\documentclass[11pt]{scrartcl} 


\usepackage{graphicx,graphics,tikz,pgfkeys}
\usetikzlibrary{arrows,decorations.pathreplacing}
\usepackage{amsmath}
\usepackage{amsthm}
\usepackage{amsfonts}
\usepackage{amssymb}
\usepackage{color}

% m.scatter(xstationAn[0],ystationAn[0],s=10,marker='o',color="#47D147",edgecolor="#004B19",zorder=4)
% m.scatter(xstationAn[1],ystationAn[1],s=10,marker='o',color="#FFF700",edgecolor="#B26B00",zorder=4)
% m.scatter(xstationAn[2],ystationAn[2],s=10,marker='o',color= "#00CCFF",edgecolor="#003D52",zorder=4)
% m.scatter(xstationAn[3],ystationAn[3],s=10,marker='o',color= "#FF99FF",edgecolor="#490049",zorder=4)


\usepackage{fixltx2e} % to be able to use the command \textsubscript
\usepackage[amssymb,thinqspace,textstyle,binary,noams,derivedinbase,derived]{SIunits} % To use SI units.
% amssymb: This option redefines the amssymb command \square to get
% the desired SIunits definition of the command.  Note: When using
% this option, the amssymb command \square can not be used.

%thinqspace This mode provides the use of \, (thin math space) as spacing be-
%        tween numerical quantities and units.

% textstyle:  When using the option textstyle units are printed in the typeface of the
%enclosing text, automatically.

%binary   This option loads the file binary.sty, which defines prefixes for binary
%         multiples.
%noams This option redefines the \micro command; use it when you don’t have
%         the AMS font, eurm10.
%derivedinbase This mode provides the ready-to-use expressions of SI derived units
%         in SI base units, e. g. \pascalbase to get ‘m−1 kg s−2 ’.
%derived This mode provides the ready-to-use expressions of SI derived units in SI
%         derived units, e. g. \derpascal to get ‘N m−2 ’.


\definecolor{darkred}{RGB}{121,27,27}
\definecolor{lightblue}{RGB}{63,151,209}
\definecolor{S}{RGB}{255,0,0}
\definecolor{N}{RGB}{0,0,255}
%

%\usetikzlibrary{arrows,decorations.pathmorphing,backgrounds,placments,fit}
\usepackage[graphics,tightpage,active]{preview}
\PreviewEnvironment{tikzpicture}
\newlength\imagewidth
\newlength\imagescale

\begin{document}

\pgfmathsetlength{\imagewidth}{20cm} % desired displayed width of image
\pgfmathsetlength{\imagescale}{\imagewidth/1200} % pixel width of image
% adjust scale of tikzpicture (and direction of y) such that pixel
% coordinates can be used for drawing overlays:


\begin{center}
\begin{tikzpicture}[font=\sffamily]


\begin{scope}[xshift=3cm, yshift=-0cm]


\begin{scope}[yshift=-2cm,xshift=-11cm]
\node[inner sep=0,outer sep=0,anchor=west] at (0cm,0cm) {\texttt{
5' \textcolor{lightblue}{CAAGCAGAAGACGGCATACGAGAT}\textcolor{darkred}{CGTGAT}\textcolor{green}{GTGACTGGAGTTCAGACGTGTGCTCTTCCGATCT}\textcolor{orange}{NNNNNN}\textcolor{blue}{\textbf{A}}\textcolor{green}{\textbf{GATCGGAAGAGC}}\textcolor{green}{GTCGTGTAGGGAAAGAGTGT}\textcolor{lightblue}{AGATCTCGGTGGTCGCCGTATCATT}
}3'};
\end{scope}

\begin{scope}[yshift=-2.5cm,xshift=-11cm]
\node[inner sep=0,outer sep=0,anchor=west] at (0cm,0cm) {\texttt{
3' \textcolor{lightblue}{GTTCGTCTTCTGCCGTATGCTCTA}\textcolor{darkred}{GCACTA}\textcolor{green}{CACTGACCTCAAGTCTGCACA}\textcolor{green}{\textbf{CGAGAAGGCTAG}}\textcolor{blue}{\textbf{A}}\textcolor{orange}{NNNNNN}\textcolor{green}{TCTAGCCTTCTCG}\textcolor{green}{CAGCACATCCCTTTCTCACA}\textcolor{lightblue}{TCTAGAGCCACCAGCGGCATAGTAA}
}5'};
\end{scope}


\end{scope}











\begin{scope}[yshift=-8.5cm]




\begin{scope}[xshift=-3.5cm,yshift=7cm]
\node[inner sep=0,outer sep=0,anchor=west] at (-3cm,-2cm) {\textcolor{green}{Read primer}};
\node[inner sep=0,outer sep=0,anchor=west] at (-3cm,-2.4cm) {\textcolor{orange}{Insert}};
\node[inner sep=0,outer sep=0,anchor=west] at (-3cm,-2.8cm) {\textcolor{blue}{3' Adenylation}};
\node[inner sep=0,outer sep=0,anchor=west] at (-3cm,-3.2cm) {\textcolor{darkred}{Index barcode}};
\node[inner sep=0,outer sep=0,anchor=west] at (-3cm,-3.6cm) {\textcolor{lightblue}{flowcell binding site}};
\end{scope}

\end{scope}



\end{tikzpicture}


\end{center}

\end{document}
